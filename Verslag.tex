\documentclass{article}
\usepackage[a4paper]{geometry}
\usepackage{parskip}
\usepackage{amsmath, amssymb}
\usepackage{ wasysym }
\usepackage{color}
\usepackage{graphicx}
\usepackage{enumerate}
\usepackage[dutch,english]{babel}
\usepackage {bm}
\usepackage{wasysym}
\usepackage{listings}
\usepackage{float}
\usepackage{epstopdf}

%\newcommand{\f}{\(f\)}
%\newcommand{\g}{\(g\)}
\newcommand{\Q}{\mathbb{Q}}
%\newcommand{\Z}{\mathbb{Z}}
%\newcommand{\R}{\mathbb{R}}
%\newcommand{\C}{\mathbb{C}}
%\newcommand{\N}{\mathbb{N}}
\newcommand{\QED}{\hfill\ensuremath{\square}}
\newcommand{\rk}{\text{rk}~}
\newcommand{\Dt}{\Delta t}

\definecolor{codegreen}{rgb}{0,0.6,0}
\definecolor{codegray}{rgb}{0.5,0.5,0.5}
\definecolor{codepurple}{rgb}{0.58,0,0.82}
\definecolor{backcolour}{rgb}{0.95,0.95,0.92}
 
\lstdefinestyle{mystyle}{
    backgroundcolor=\color{backcolour},   
    commentstyle=\color{codegreen},
    keywordstyle=\color{magenta},
    numberstyle=\tiny\color{codegray},
    stringstyle=\color{codepurple},
    basicstyle=\footnotesize,
    breakatwhitespace=false,         
    breaklines=true,                 
    captionpos=b,                    
    keepspaces=true,                 
    numbers=left,                    
    numbersep=5pt,                  
    showspaces=false,                
    showstringspaces=false,
    showtabs=false,                  
    tabsize=2
}
\lstset{style=mystyle}
\title{Numerieke Methoden 1 Practicum Traffic Flow}

\author{Casper Barendrecht \& Stijn Moerman\\ s1693441 \& s1696874}

\date{\today}


\begin{document}
\maketitle



\textbf{Opgave 1}\\
Laat \(u(x,t)=\alpha\rho + \beta\).

Er geldt:
\begin{align}
	\frac{\partial \rho}{\partial t}+\frac{\partial}{\partial x}(\rho(1-\rho))=\nu \frac{\partial^2 \rho}{\partial x^2},~\nu\in \mathbb{R}_+ \label{eq:rho}\\
 \frac{\partial u}{\partial t}+\frac{\partial}{\partial x}\left(\frac{1}{2}u^2\right) = \nu \frac{\partial^2 u}{\partial x^2},~\nu\in \mathbb{R}_+\label{eq:u}
\end{align}
derp
\begin{align*}
 \frac{\partial u}{\partial t} &=\alpha\frac{\partial \rho}{\partial t}\\
\frac{\partial}{\partial x}\left(\frac{1}{2}u^2\right)&=\frac{1}{2}\frac{\partial}{\partial x}(\alpha^2\rho^2 +2\alpha\beta\rho+\beta^2)= \frac{\alpha}{2}\frac{\partial}{\partial x}\rho(\alpha\rho+2\beta))\\
\nu \frac{\partial^2 u}{\partial x^2}&=\alpha \nu\frac{\partial^2 \rho}{\partial x^2}
\end{align*}
En dus:
\begin{align*}
	\alpha\frac{\partial \rho}{\partial t}+\frac{\alpha}{2}\frac{\partial}{\partial x}\rho(\alpha\rho+2\beta))=\alpha \nu\frac{\partial^2 \rho}{\partial x^2}\\
		\frac{\partial \rho}{\partial t}+\frac{\partial}{\partial x}\rho(\beta-(-\frac{\alpha}{2})\rho)=\nu\frac{\partial^2 \rho}{\partial x^2}
\end{align*}
We vinden dat verglijking \eqref{eq:rho} en \eqref{eq:u} hetzelfde zijn als \(\beta=1\) en \(\alpha=-2\). We vinden dat \(u(x,t)=-2\rho+\beta\).\\

\textbf{Opgave 3}
Mooie tabel is mooi:

\begin{table}[h!]
\centering

\label{tab:specs}
\begin{tabular}{|l|l|l|l|l|}
\hline
\(v\) & \(N\) & \(L\) & \(t_e\) & \(\Dt\) \\ \hline
 0.5 & 100 & 3.0 & 5.0 & 0.0001 \\ \hline
\end{tabular}
\caption{Startvoorwaarden}
\end{table}
\textbf{Opgave 6}
Laat nu: 
\begin{align*}
	u^0(x) =\begin{cases}
	1, & 0\leq x \leq L/3,\\
	2-(3/L)x, & L/3 \leq x\leq 2L/3,\\
	0, & 2L/3 \leq x
	\end{cases}
\end{align*}
samen met de nieuwe randvoorwaarden 
\[u_l(t)=1,~u_r(t)=0,~t\geq 0\]
en de nieuwe parameters:
\begin{table}[h!]
\centering

\label{tab:specs2}
\begin{tabular}{|l|l|l|l|l|}
\hline
\(v\) & \(N\) & \(L\) & \(t_e\) & \(\Dt\) \\ \hline
 0.01 & 100 & 3.0 & 5.0 & 0.003 \\ \hline
\end{tabular}
\caption{Startvoorwaarden}
\end{table}


\end{document}
%plaatje opgave 5 is op t=0.3600
