\documentclass{article}
\usepackage[a4paper]{geometry}
\usepackage{parskip}
\usepackage{amsmath, amssymb}
\usepackage{ wasysym }
\usepackage{color}
\usepackage{graphicx}
\usepackage{enumerate}
\usepackage[dutch,english]{babel}
\usepackage {bm}
\usepackage{wasysym}
\usepackage{listings}
\usepackage{float}
\usepackage{epstopdf}

%\newcommand{\f}{\(f\)}
%\newcommand{\g}{\(g\)}
\newcommand{\Q}{\mathbb{Q}}
%\newcommand{\Z}{\mathbb{Z}}
%\newcommand{\R}{\mathbb{R}}
%\newcommand{\C}{\mathbb{C}}
%\newcommand{\N}{\mathbb{N}}
\newcommand{\QED}{\hfill\ensuremath{\square}}
\newcommand{\rk}{\text{rk}~}
\newcommand{\Dt}{\Delta t}

\definecolor{codegreen}{rgb}{0,0.6,0}
\definecolor{codegray}{rgb}{0.5,0.5,0.5}
\definecolor{codepurple}{rgb}{0.58,0,0.82}
\definecolor{backcolour}{rgb}{0.95,0.95,0.92}
 
\lstdefinestyle{mystyle}{
    backgroundcolor=\color{backcolour},   
    commentstyle=\color{codegreen},
    keywordstyle=\color{magenta},
    numberstyle=\tiny\color{codegray},
    stringstyle=\color{codepurple},
    basicstyle=\footnotesize,
    breakatwhitespace=false,         
    breaklines=true,                 
    captionpos=b,                    
    keepspaces=true,                 
    numbers=left,                    
    numbersep=5pt,                  
    showspaces=false,                
    showstringspaces=false,
    showtabs=false,                  
    tabsize=2
}
\lstset{style=mystyle}
\title{Numerieke Methoden 1 Practicum Traffic Flow}

\author{Casper Barendrecht \& Stijn Moerman\\ s1693441 \& s1696874}

\date{\today}


\begin{document}
\maketitle



\textbf{Opgave 1}\\
Laat \(u(x,t)=\alpha\rho + \beta\).

Er geldt:
\begin{align}
	\frac{\partial \rho}{\partial t}+\frac{\partial}{\partial x}(\rho(1-\rho))=\nu \frac{\partial^2 \rho}{\partial x^2},~\nu\in \mathbb{R}_+ \\
 \frac{\partial u}{\partial t}+\frac{\partial}{\partial x}\left(\frac{1}{2}u^2\right) = \nu \frac{\partial^2 u}{\partial x^2},~\nu\in \mathbb{R}_+
\end{align}
derp
\begin{align*}
 \frac{\partial u}{\partial t} &=\alpha\frac{\partial \rho}{\partial t}\\
\frac{\partial}{\partial x}\left(\frac{1}{2}u^2\right)&=\frac{1}{2}\frac{\partial}{\partial x}(\alpha^2\rho^2 +2\alpha\beta\rho+\beta^2)= \frac{1}{2}(\alpha^2+2\alpha\beta)\frac{\partial \rho}{\partial x}\\
\nu \frac{\partial^2 u}{\partial x^2}&=\alpha \nu\frac{\partial^2 \rho}{\partial x^2}
\end{align*}
En dus:
\begin{align*}
	\alpha\frac{\partial \rho}{\partial t}+\frac{1}{2}(\alpha^2+2\alpha\beta)\frac{\partial \rho}{\partial x}=\alpha \nu\frac{\partial^2 \rho}{\partial x^2}
\end{align*}
%We vinden \(\alpha=1\).

\end{document}
